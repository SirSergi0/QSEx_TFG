\documentclass[12pt,letterpaper]{article}
\usepackage{fullpage}
\usepackage[top=1.5cm, bottom=3.5cm, left=2.2cm, right=2.2cm]{geometry}
\usepackage{amsmath,amsthm,amsfonts,amssymb,amscd, esint}
\usepackage{lastpage}
\usepackage{enumerate}
\usepackage{fancyhdr}
\usepackage{mathrsfs}
\usepackage{graphicx}
\usepackage{listings}
\usepackage{hyperref}
\usepackage[english]{babel}
\usepackage{lipsum}
\usepackage[table,xcdraw]{xcolor}
\usepackage{enumitem}
\usepackage{float}
\usepackage{chemfig}
\usepackage{yfonts}
\usepackage{braket}
\usepackage{dsfont}
\usepackage{tikz}
\usepackage{wrapfig}
\usepackage{url}
\usepackage{natbib}
\usepackage[normalem]{ulem}
\usepackage{multicol}
\useunder{\uline}{\ul}{}


%%%%%%%%%%%%%%% CODELISTINGS %%%%%%%%%%%%%%%%
\usepackage{listings}
\definecolor{codegreen}{rgb}{0,0.6,0}
\definecolor{codegray}{rgb}{0.5,0.5,0.5}
\definecolor{codepurple}{rgb}{0.58,0,0.82}
\definecolor{backcolour}{rgb}{0.95,0.95,0.92}

\lstdefinestyle{mystyle}{
    backgroundcolor=\color{backcolour},   
    commentstyle=\color{codegreen},
    keywordstyle=\color{magenta},
    numberstyle=\tiny\color{codegray},
    stringstyle=\color{codepurple},
    basicstyle=\ttfamily\footnotesize,
    breakatwhitespace=false,
    breaklines=true,                 
    captionpos=t,                    
    keepspaces=true,                 
    numbers=left,                    
    numbersep=5pt,                  
    showspaces=false,                
    showstringspaces=false,
    showtabs=false,                  
    tabsize=2
}

\lstset{style=mystyle}

%%%%%%%%%%%%%%%%%%%%%%%%%%%%%%%%%%%%%%% 
\usepackage{titlesec}
\usepackage{textcase} % for uppercase handling

% --- Section formatting ---
\titleformat{\section}
  {\normalfont\normalsize\bfseries\centering} % font size, bold, centered
  {\thesection}{1em}{\MakeTextUppercase} % uppercase text

% --- Subsection formatting ---
\titleformat{\subsection}
  {\normalfont\normalsize\bfseries\centering}
  {\thesubsection}{1em}{\MakeTextUppercase}

% --- Roman numerals for numbering ---
\renewcommand{\thesection}{\Roman{section}}
\renewcommand{\thesubsection}{\Roman{section}.\Roman{subsection}}

\newtheorem{definition}{Definition}
\newtheorem{observation}{Observation}
\newtheorem{reflection}{Reflection}
\newtheorem{PyPackage}{Package}
\newtheorem{book}{Book}

\newcommand{\HRule}[1]{\rule{\linewidth}{#1}}
\setcounter{tocdepth}{5}
\setcounter{secnumdepth}{5}

%\setlength{\parindent}{0.0in}
%\setlength{\parskip}{0.05in}

% Edit these as appropriate
\newcommand\course{}
\newcommand\subject{Final Degree Project}
\newcommand\degree{Bachelor's Degree in Physics}
\newcommand\documenttitle{Lower bounds of the success probability in quantum state exclusion for general ensembles}
\newcommand\NetIDb{Universitat Autònoma de Barcelona}
\newcommand\AuthorName{Sergio Castañeiras Morales}

\hypersetup{%
  pdftitle  = \documenttitle,
  pdfauthor = \AuthorName,
  pdfsubject= \degree,
  pdfcreator= \AuthorName,
  hidelinks = true,
}

\usepackage{glossaries}
\usepackage{glossary-longragged}

\makenoidxglossaries
\newacronym{qse}{QSE}{Quantum State Exclusion}
\newacronym{qsd}{QSD}{Quantum State Discrimination}
\newacronym{sdp}{SDP}{Semidefinite Program}
\newacronym{povm}{POVM}{Positive Operator-Valued Measure}
\newacronym{me}{ME}{Minimum Error}
\newacronym{ze}{ZE}{Zero Error}
\glsaddall[types=\acronymtype] 

\DeclareMathOperator{\tr}{Tr}
\usetikzlibrary{arrows.meta, positioning}

\begin{document}
\title{\vspace{4cm} \normalsize 
		\includegraphics[width = 0.25\textwidth]{GeneralSources/UABLogo.png}\\ [0.5cm]
		\textsc{\NetIDb}\\ [2.0cm]
		\HRule{0.5pt} \\
		\LARGE \textbf{\uppercase{\documenttitle}}
		\HRule{2pt} \\ [1.5cm]
		\normalsize \begin{tabular}{rcl}  % Create a right-left column alignment
        \textsc{Author} & : & \textsc{\AuthorName} \\
        \textsc{Supervisor} & : & \textsc{Ramón Muñoz Tapia} \\
        \textsc{Co-Supervisor} & : & \textsc{Santiago Llorens Fernández}
    \end{tabular}
    \normalsize \vspace*{5\baselineskip}
		}

\date{2024-2025}

\author{\large \textsc{\subject} \\ \textsc{\degree}}



\begin{titlepage}
\clearpage\maketitle
\thispagestyle{empty}
\end{titlepage}

\thispagestyle{empty}
\mbox{} 
\newpage
\thispagestyle{empty}
\vspace*{\fill} % Push content to vertical center
\begin{flushright}
    \emph{Ab ovo usque ad mala.}\\[1em]
    \textbf{Horace}
\end{flushright}
\vspace*{\fill} 
\newpage

\newpage
\pagestyle{fancyplain}
\headheight 35pt
\lhead{\NetIDb}    
\rhead{\subject}
\cfoot{\small\thepage}
\headsep 1.5em
\setcounter{page}{1}
\begin{multicols}{2}[\begin{center}
\begin{abstract}
Given a quantum state known to be prepared from an ensemble of two or more states, quantum state exclusion aims to rule out the possibility that it was prepared in a particular state from the ensemble. Using the known solution for group generated ensembles \cite{MainPaper}, we study this result as a lower bound for randomly generated ensembles via semidefinite programming.
\end{abstract}
\end{center}
Keywords: \emph{SDP, Quantum state exclusion , \textcolor{blue}{Add keywords}}.
]
\section{Introduction}
Adri guapo

In many real-world scenarios, excluding a certain hypothesis can be more practical than solving the problem entirely. For instance, in disease diagnosis, ruling out potential diseases often serves as the first step in identifying the actual condition. Similarly, when repairing a machine, it is sometimes more efficient to identify the components that are functioning correctly, which narrows down the search for the faulty part.

In this project, we project this idea into the quantum realm by focusing on \gls{qse}. Rather than determining the exact state of a quantum system, we aim to eliminate one or more possible candidates from a known ensemble of states. Notice, this approach can be more suitable or efficient in certain quantum information tasks.

Given a quantum state known to be prepared from a finite ensemble, \gls{qsd} seeks to identify which specific state from the ensemble corresponds to the given system. In contrast, \gls{qse}~\cite{PhysicsExclusionSource} adopts the opposite perspective: it aims to determine which states from the ensemble do not correspond to the prepared state. While \gls{qsd} has been deeply studied in recent years\cite{DiscriminationArticle}, with significant advances since its inception~\cite{helstromBook}, \gls{qse} offers a complementary framework with distinct advantages.

Although the tasks of exclusion and discrimination coincide for ensembles containing only two states \footnote{Since for the two states case excluding one necessarily implies identifying the other.}, when dealing with ensembles of three or more states, the two problems diverge in both approach and complexity. One of the most significant features of \gls{qse} is the possibility of achieving \emph{perfect exclusion}, where certain states can be ruled out with zero probability of error in cases where \emph{perfect discrimination} is impossible\cite{OptimalitySRM}. 

This capability opens new frontiers in quantum information theory, particularly in the context of partial information retrieval from quantum systems. By excluding certain states, it is possible to gain insight into the encoded information without needing to fully determine the original state.

As with \gls{qsd}, obtaining a general analytical solution for \gls{qse} remains an open problem. However, analytical results have been found in specific cases when the ensemble of quantum states exhibits a certain degree of symmetry. In particular, when the ensemble is generated by the action of a finite group, the problem becomes more tractable and exact solutions have been derived.

The exclusion task can be carried out under two main protocols: \gls{me} and \gls{ze}\footnote{Also known as \emph{unambiguous exclusion}.}. In the Minimum Error scenario, the goal is to minimize the probability of mistakenly excluding the actual prepared state. In contrast, the Zero Error approach seeks to exclude a state with absolute certainty, even if that means sometimes the procedure yields an inconclusive result.

Building on recent results that provide exact solutions for exclusion tasks in group generated ensembles \cite{MainPaper}, this project undertakes a numerical study of such results as lower bounds for more general, randomly generated ensembles. To this end, we employ \gls{sdp} to explore \gls{qse} performance in arbitrary settings. Furthermore, we investigate improved bounds for the general case based on how closely a given ensemble resembles a group generated one\footnote{The notion of "how close" will be formally defined in Section~\textcolor{blue}{add section}.}.

\subsection{Formulation of the Problem}

Let $\left\{(\rho_i, \eta_i)\right\}_{i=1}^n$ be an ensemble of $n$ quantum states, where each $\rho_i$ denotes a pure state density matrix\footnote{This formulation hols true for mixed states but the project will only discuss the pure state scenario.}, i.e., $\rho_i = \ket{\psi_i}\bra{\psi_i}$, and $\eta_i$ represents the prior probability of occurrence of the state $\rho_i$. Let $\rho_j$ be the target state from this ensemble. Our objective is to develop a procedure to identify another state $\rho_k \in \left\{\rho_i\right\}_{i=1}^n$, such that $\rho_k \neq \rho_j$.

Quantum measurements are described by a set of \glspl{povm}, denoted by $\left\{\Pi_i\right\}_{i=1}^n$, acting on the Hilbert space $\mathcal{H}$ of the quantum system. Here we pewsent the two studied protocols for \gls{qse}: minimum-error (\gls{me}) and zero-error (\gls{ze}).

The goal of the \gls{me} protocol is to minimize the probability of incorrectly excluding the target state from our hypothesis. If we formulate it as an \gls{sdp}, the problem reads,\footnote{Note that the \gls{sdp} formulations of quantum state discrimination may differ from the exclusion ones by interchanging minimization and maximization problems.}
\begin{align*}
	P_{\text{\gls{me}}}^e = \min_{\left\{\Pi_i\right\}} &\sum_{i=1}^n \tr(\Pi_i \rho_i),\\
	\text{subject to} \quad & \sum_{i=1}^n \Pi_i = \mathds{1}, \quad \Pi_i \geq 0 \quad \forall i \in \{1, \dots, n\}.
\end{align*}

Note the constraints $\sum_{i=1}^n \Pi_i = \mathds{1}$ and $\Pi_i \geq 0$ ensure that the $\Pi_i$ form a valid POVM, since they demand positive semi-definition and form a complete measurement. The superscript $e$ in $P_{\text{\gls{me}}}^e$ indicates that this is the \emph{error probability}.

Alternatively, we may formulate the problem in terms of the \emph{success probability}, denoted by $P_{\text{\gls{me}}}^s$, which quantifies the probability of a correct exclusion. This equivalent formulation reads,
\begin{align*}
	P_{\text{\gls{me}}}^s = \max_{\left\{\Pi_i\right\}} &\left(1 - \sum_{i=1}^n \tr(\Pi_i \rho_i)\right),\\
	\text{subject to} \quad & \sum_{i=1}^n \Pi_i = \mathds{1}, \quad \Pi_i \geq 0 \quad \forall i \in \{1, \dots, n\}.
\end{align*}

Naturally, both formulations are related via:
\begin{align*}
P_{\text{\gls{me}}}^s + P_{\text{\gls{me}}}^e = 1.
\end{align*}

In the case of the \gls{ze} protocol, the POVMs must also satisfy an unambiguity condition, i.e. each measurement operator $\Pi_i$ must be orthogonal to the corresponding state $\rho_i$. In other words,
\begin{align*}
\tr(\Pi_i \rho_i) = 0 \quad \forall i \in \{1, \dots, n\}.
\end{align*}

To ensure completeness, we introduce an additional POVM element $\Pi_?$ representing an inconclusive result,
\begin{align*}
\Pi_? = \mathds{1} - \sum_{i=1}^n \Pi_i.
\end{align*}
If the measurement yields the outcome $\Pi_?$ (i.e., the "?" symbol), the result is inconclusive.

The corresponding \gls{sdp} for minimizing the probability of an inconclusive result (i.e., error) in the \gls{ze} protocol is:
\begin{align*}
	P_{\text{\gls{ze}}}^e = \min_{\left\{\Pi_i\right\}} &\sum_{i=1}^n \tr(\Pi_? \rho_i),\\
	\text{subject to} \quad & \sum_{i=1}^n \Pi_i + \Pi_? = \mathds{1}, \quad \Pi_? \geq 0,\\
	& \tr(\Pi_i \rho_i) = 0, \quad \Pi_i \geq 0 \quad \forall i.
\end{align*}

The corresponding success probability is naturally given by,
\begin{align*}
	P_{\text{\gls{ze}}}^s = \max_{\left\{\Pi_i\right\}} &\left(1 - \sum_{i=1}^n \tr(\Pi_? \rho_i)\right),\\
	\text{subject to} \quad & \sum_{i=1}^n \Pi_i + \Pi_? = \mathds{1}, \quad \Pi_? \geq 0,\\
	& \tr(\Pi_i \rho_i) = 0, \quad \Pi_i \geq 0 \quad \forall i.
\end{align*}

This formulation is analogous to the \gls{me} protocol, with the crucial difference being the constraint $\tr(\Pi_i \rho_i) = 0$, enforcing unambiguous discrimination.

\subsection{Gram matrix formulation}
Let $\mathcal{G} \in \mathbb{C}^{n \times n}$ be the \emph{Gram matrix} of the system, defined as the $n\times n$ positive semidefinite hermitian matrix such that,
\begin{equation*}
	\mathcal{G} =
	\begin{pmatrix}
		\braket{\psi_1|\psi_1} & \braket{\psi_1|\psi_2} & \dots & \braket{\psi_1|\psi_n}\\
		\braket{\psi_2|\psi_1} & \braket{\psi_2|\psi_2} & \dots & \braket{\psi_2|\psi_n}\\
		\vdots & \vdots & \ddots & \vdots\\
		\braket{\psi_n|\psi_1} & \braket{\psi_n|\psi_2} & \dots & \braket{\psi_n|\psi_n}
	\end{pmatrix},
\end{equation*}
i.e., $\mathcal{G}_{i,j} = \braket{\psi_i|\psi_j}$. Since all states are normalized, we have,
\begin{equation*}
	\mathcal{G} =
	\begin{pmatrix}
		1 & \braket{\psi_1|\psi_2} & \dots & \braket{\psi_1|\psi_n}\\
		\braket{\psi_2|\psi_1} & 1 & \dots & \braket{\psi_2|\psi_n}\\
		\vdots & \vdots & \ddots & \vdots\\
		\braket{\psi_n|\psi_1} & \braket{\psi_n|\psi_2} & \dots & 1
	\end{pmatrix}.
\end{equation*}

Notice this matrix is Hermitian by construction,
\begin{align*}
\mathcal{G}_{i,j}^* = (\braket{\psi_i|\psi_j})^* = \braket{\psi_j|\psi_i} = \mathcal{G}_{j,i}.
\end{align*}

The Gram matrix allows us to reframe the exclusion problem in a more abstract and basis-independent form. Since $\mathcal{G}$ is hermitian and positive semi-definite, we can write,
\begin{align*}
\mathcal{G} = X^\dagger X,
\end{align*}
for some matrix $X$ whose columns are the pure states,
\begin{align*}
	X = \begin{pmatrix}
		\mid & \mid &        & \mid \\
		\ket{\psi_1} & \ket{\psi_2} & \dots & \ket{\psi_n} \\
		\mid & \mid &        & \mid
	\end{pmatrix}.
\end{align*}
Notice the diagonal elements of $X$ are,
\begin{align*}
	X_{i,i}=\braket{\omega_i|\psi_i}
\end{align*}
For an arbitrary orthonormal basis $\left\{\ket{\omega_i}\right\}_{i=1}^n$. \footnote{It is important to remark that fixing the basis $\left\{\ket{\omega_i}\right\}_{i=1}^n$ fixes the decomposition of $G=X^\dagger X$ and vice versa.}

Let us consider an arbitrary orthonormal basis $\left\{\ket{\omega_i}\right\}_{i=1}^n$, and define the POVM elements as projectors $\Pi_i = \ket{\omega_i}\bra{\omega_i}$. Then,
\begin{align*}
	\tr(\Pi_i \rho_i) &= \tr\left( \ket{\omega_i}\bra{\omega_i} \ket{\psi_i}\bra{\psi_i} \right) \\
	&= |\braket{\omega_i|\psi_i}|^2 \\
	&= |X_{i,i}|^2.
\end{align*}

Therefore, we can reformulate the \gls{sdp} for the \gls{me} protocol as,
\begin{align*}
	P_{\gls{me}}^e = \min_{X} &\sum_{i=1}^n |X_{i,i}|^2,\\
	\text{subject to} \quad & X^\dagger X = \mathcal{G}, \quad X \geq 0.
\end{align*}

Similarly, the success probability becomes, 
\begin{align*}
	P_{\gls{me}}^s = \max_{X} &\left(1 - \sum_{i=1}^n |X_{i,i}|^2\right),\\
	\text{subject to} \quad & X^\dagger X = \mathcal{G}, \quad X \geq 0.
\end{align*}

This reformulation highlights that if two ensembles $A$ and $B$ share the same Gram matrix, then their exclusion problems, in both \gls{me} and \gls{ze} protocols, are equivalent. That is, the optimal success or error probabilities are identical in both systems. Hence, we focus on the Gram matrix to analyze exclusion problems, rather than relying on explicit state representations.

\subsection{Group Generated Ensembles}

Given a quantum state $\ket{\psi}$, which we refer to as the \emph{seed state}, we define a \emph{group generated ensemble} as the set of states obtained by applying a group of unitary transformations to $\ket{\psi}$. Specifically, if the ensemble consists of a total of $n$ quantum states, then its elements are of the form,
\begin{align*}
	U_i\ket{\psi}, \quad i \in \{1, \dots, n\},
\end{align*}
where the set of unitary matrices $\{U_i\}_{i=1}^n$ forms a finite group under standard matrix product. In terms of density matrices, the ensemble can equivalently be written as:
\begin{align*}
	\rho_i = U_i \rho U_i^\dagger, \quad i \in \{1, \dots, n\},
\end{align*}
where $\rho = \ket{\psi}\bra{\psi}$ is the density matrix corresponding to the seed state.

For instance, let $U$ be a unitary operator such that $U^n = \mathds{1}$ (i.e., $U$ generates a cyclic group of order $n$). Then, the set of states
\begin{align*}
	\left\{ U^i\ket{\psi} \right\}_{i=0}^{n-1}
\end{align*}
forms a group generated ensemble based on the cyclic group $\mathbb{Z}/n\mathbb{Z}$. This type of ensemble is of particular interest in our study and will be explored in more detail in subsequent sections.

Let $U \in SU(n)$ be an $n \times n$ special unitary matrix satisfying $U^n = \mathds{1}$, and let $\ket{\psi}$ be the seed state. The Gram matrix $G$ associated with the ensemble $\left\{U^i\ket{\psi}\right\}_{i=0}^{n-1}$ is nothing but,
\begin{align*}
	\mathcal{G}_{ij} = \braket{U^i\psi | U^j\psi} = \braket{\psi | U^{j-i} | \psi} = \braket{\mathcal{U}^{j-i}}_{\psi},
\end{align*}
where we use the shorthand notation
\begin{align*}
\braket{\mathcal{U}^k}_\psi := \braket{\psi | U^k | \psi}.
\end{align*}

Using this, the Gram matrix $G$ can be expressed as a circulant matrix:
\begin{align*}
	\mathcal{G} = \begin{pmatrix}
 1 & \braket{\mathcal{U}}_{\psi} & \braket{\mathcal{U}^2}_{\psi} & \cdots & \braket{\mathcal{U}^{n-1}}_{\psi} \\
 \braket{\mathcal{U}^{n-1}}_{\psi}^* & 1 & \braket{\mathcal{U}}_{\psi} & \cdots & \braket{\mathcal{U}^{n-2}}_{\psi} \\
 \braket{\mathcal{U}^{n-2}}_{\psi}^* & \braket{\mathcal{U}^{n-1}}_{\psi}^* & 1 & \cdots & \braket{\mathcal{U}^{n-3}}_{\psi} \\
 \vdots & \vdots & \vdots & \ddots & \vdots \\
 \braket{\mathcal{U}}_{\psi}^* & \braket{\mathcal{U}^{2}}_{\psi}^* & \braket{\mathcal{U}^{3}}_{\psi}^* & \cdots & 1
\end{pmatrix}.
\end{align*}

Note the Gram matrix is Hermitian, as required, since,
\begin{align*}
    \braket{\mathcal{U}^{-j}}_{\psi} = \braket{\psi | U^{-j} | \psi} = \left(\braket{\psi | U^j | \psi}\right)^* = \braket{\mathcal{U}^j}_{\psi}^*.
\end{align*}
Additionally, by using the identity $U^n = \mathds{1}$, we can simplify terms such as,
\begin{align*}
    \braket{\mathcal{U}^{-n+i}}_\psi = \braket{\psi | U^{-n+i} | \psi} = \braket{\psi | U^i | \psi} = \braket{\mathcal{U}^i}_\psi.
\end{align*}
Therefore we may write,
\begin{align}
	\mathcal{G}=\begin{pmatrix}
 1 & \braket{\mathcal{U}}_{\psi} & \braket{\mathcal{U}^{2}}_{\psi} & \hdots &  \braket{\mathcal{U}}_{\psi}^* \\
  \braket{\mathcal{U}}_{\psi}^* & 1 & \braket{\mathcal{U}}_{\psi} & \hdots &  \braket{\mathcal{U}^{2}}_{\psi}^* \\
    \braket{\mathcal{U}^{2}}_{\psi}^* &  \braket{\mathcal{U}}_{\psi}^*  & 1 & \hdots &  \braket{\mathcal{U}^{3}}_{\psi}^* \\
   \vdots & \vdots & \vdots & \ddots & \vdots \\
  \braket{\mathcal{U}}_{\psi} & \braket{\mathcal{U}^{2}}_{\psi}  & \braket{\mathcal{U}^{3}}_{\psi}  & \hdots &  1 
\end{pmatrix}.
\end{align}

This confirms the circulant structure of $\mathcal{G}$, where each row is a cyclic permutation of the one above it. In other words the Gram matrix of $\mathbb{Z}/n\mathbb{Z}$ matrices are circulant matrixes. This result is especially useful, as they can be diagonalized by the discrete Fourier basis, which simplifies many tasks\cite{circulantMatrices}.

Moreover we can compute the Gram matrix of a group generated ensamble by generating the Cayley table, also known as the \emph{multiplication table}, of the group. For instance if we consider the smallest non-commutative group $S_3$ composed by the rotations and symmetries of the triangle. If we denote the identity as $e$, the 3 symmetries as $p$, $q$ and $r$ and $s$ as a rotation, we know the Cayley table to be,
\begin{table}[H]
	\centering
	\caption{Cayley table of the $S_3$ group.}
	\begin{tabular}{c||c c c c c c}
		$S_3$ & $e$ & $p$ & $q$ & $r$ & $s$ & $s^2$ \\\hline\hline
		$e$   & $e$ & $p$ & $q$ & $r$ & $s$ & $s^2$ \\
		$p$  & $p$ & $e$ & $s^2$ & s & $q$ & $r$ \\
		$q$  & $q$ & $s$ & $e$ & $s^2$ & $r$ & $p$ \\
		$r$  & $r$ & $s^2$ & $s$ & $e$ & $p$ & $q$ \\
		$s^2$ & $s^2$ & $r$ & $p$ & $q$ & $e$ & $s$ \\
		$s$ & $s$ & $q$ & $r$ & $p$ & $s^2$ & $e$ 
	\end{tabular}
\end{table}

\begin{table}[H]
    \centering
    \caption{Cayley table of the $S_3$ group.}
    \begin{tabular}{c||c c c c c c}
        $S_3$ & $e$ & $p$ & $q$ & $r$ & $s$ & $s^2$ \\ \hline \hline
        $e$   & $e$ & $p$ & $q$ & $r$ & $s$ & $s^2$ \\
        $p$   & $p$ & $e$ & $s^2$ & $s$ & $q$ & $r$ \\
        $q$   & $q$ & $s$ & $e$ & $s^2$ & $r$ & $p$ \\
        $r$   & $r$ & $s^2$ & $s$ & $e$ & $p$ & $q$ \\
        $s^2$ & $s^2$ & $r$ & $p$ & $q$ & $e$ & $s$ \\
        $s$   & $s$ & $q$ & $r$ & $p$ & $s^2$ & $e$
    \end{tabular}
\end{table}
where we enforced the diagonal of the table 


%%%%%%%%%%%%%%%%%%%%%%%%%%%%%%%%%%%%%%%%%%
%%%%%%%%%%%%%%%% BIBLIOGRAPHY %%%%%%%%%%%%%%%%%
%%%%%%%%%%%%%%%%%%%%%%%%%%%%%%%%%%%%%%%%%%

\bibliographystyle{plain}
\bibliography{references} 
%%%%%%%%
\section*{List of abbreviations}
\renewcommand{\glsnamefont}[1]{\textbf{#1}}
\printnoidxglossary[type=main, title={\vspace{-1cm}}, nonumberlist, nogroupskip, style=super]


\end{multicols}
\end{document}
